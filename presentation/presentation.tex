\documentclass[english,svgnames,hide notes,12pt]{beamer}
\usepackage{macros-ohp}
\usepackage{/Users/alexc/HOL/src/TeX/holtexbasic}
\usepackage{bussproofs}
\usepackage{alltt}
\usepackage{MnSymbol}
\usepackage{textgreek}
% \usepackage{thmtools}
% \usepackage{listings}             

\newtheorem{thm}{Theorem}
\theoremstyle{definition}
\newtheorem{defn}{Definition}
\theoremstyle{remark}
\newtheorem{notn}{Notation}
% \declaretheorem[style=definition,name=Definition]{defn}
% \decalretheorem[style=remark,name=Notation]{notn}

\def\presentationtitle{Proof of calculi equivalence in HOL4}

\title{\large\presentationtitle}
\author{Alexander Cox\\
		\small Supervised by Michael Norrish\\
		\small The Australian National University
	}
\date{\today}

% \AtBeginSection[]{
%   \begin{frame}
%     \frametitle{Table of Contents}
%     \tableofcontents[currentsection]
%   \end{frame}}

\begin{document}
\thispagestyle{empty}
%\begin{frame}[plain] %put [plain] at the end to get rid of the page number on this page
\begin{frame}
    \titlepage{}
\end{frame}

\begin{frame}
    \frametitle{Introduction}
    For my VRS project, I have been formalising some proof theory in an interactive theorem prover called HOL. 

    \bigskip
    The proof I have been formalising is the equivalence between Natural Deduction and Sequent Calculus, for intuitionistic propositional logic.

    \bigskip
    This result comes from \emph{proof theory},  a branch of mathematical logic that analyses proofs and their calculi. 
\end{frame}

\section{Theorem Proving in HOL}

\begin{frame}
    \frametitle{Theorem Provers}
    \begin{itemize}
        \item An \emph{Interactive Theorem Prover} or \emph{Proof Assistant} is software used to formalise and verify the correctness of proofs.
        \item The prover I used (HOL4) is engineered so that theorems can only be produced under the control of a small trusted kernel.
        \item Proving a proposition in a proof assistant usually takes significantly more time than proving an informal proof on paper.
    \end{itemize}
\end{frame}

\begin{frame}
    \frametitle{Why Bother?}
    \begin{enumerate}
        \item Once a theorem is proved in a proof assistant, you can trust that it is a correct proof, provided that the proof assistant itself is sound.
        \item Proving a theorem formally can reveal problems with the informal proof in some cases, since details cannot be skimmed over like they sometimes are in informal proofs.
        \item I wanted to learn about theorem provers, and this seemed like a good project for that.
        % \item It is also fun (in my opinion).
    \end{enumerate}
\end{frame}

% \begin{frame}
% 	\frametitle{The HOL Theorem Prover}
%     \begin{itemize}
%         \item The HOL Theorem Prover is a proof assistant developed since the mid-80s. 
%         \item The version I have been using is called HOL4. 
%         \item HOL implements a Higher Order Logic which is a variant of Church's simple theory of types.
%         \item HOL is written in Standard ML (SML), which is also the language the user interacts with.
%     \end{itemize}
% \end{frame}

% \begin{frame}
%     \frametitle{Theorems in HOL}
%     \begin{itemize}
%         \item Theorems in HOL are represented as an SML datatype, and the only way to produce them is to use sound inference rules on predefined axioms.
%         \item As a consequence, you can't prove something which is false, unless there is a fault with an axiom or inference rule.
%         \item HOL is heavily scrutinised by experts in the field, and is generally trusted to be sound.
%     \end{itemize}
% \end{frame}

% \begin{frame}
% 	\frametitle{Goal-directed Proof}
%     \begin{itemize}
%         \item When proving something in HOL, I have been using goal-directed proofs.
%         \item I state the goal I want to prove, and try to prove it using \emph{tactics}. Some tactics are very powerful and can prove simple goals without much help at all. Others goals require more work, often splitting into multiple sub-goals. 
%         \item Generally I work forwards (from assumptions) and backwards (from the conclusion/goal) at the same time.
%     \end{itemize}
% \end{frame}

\section{Proof Theory and Logic}

\begin{frame}[fragile]
    % \frametitle{Propositional Logic Syntax}
    Proofs manipulate logic formulae, which are defined inductively:
    \[ \varphi ::= a ~|~ \varphi \lor \varphi ~|~ \varphi \land \varphi ~|~ \varphi \to \varphi ~|~ \bot \]

    \bigskip
    In HOL:
    \small
    \begin{alltt}
        val _ = Datatype `formula =
            Var 'a
            | Or formula formula
            | And formula formula
            | Imp formula formula
            | Bot`;
    \end{alltt}
\end{frame}
% \begin{frame}
%     % \frametitle{Propositional Logic Syntax (cont.)}
%     \begin{notn}
%     Some abbreviations: 
%         \[ \neg \varphi := \varphi \to \bot \]
%         \[ \varphi \Leftrightarrow \psi := \varphi \to \psi \land \psi \to \varphi \]
%         \[ \top := \bot \to \bot \]
%         In HOL:
%         \begin{alltt}
%             val Not_def = Define `Not f = f Imp Bot`;\\
%             val BiImp_def = \\
%             ~Define `f BiImp f' = (f Imp f') And (f' Imp f)`;\\
%             val Top_def = Define `Top = Bot Imp Bot`;\\
%         \end{alltt}
%     \end{notn}
% \end{frame}
\newcommand{\bs}{\char`\\}
\begin{frame}[fragile]
    \frametitle{Inference Rules}
    An inference rule in a system $S$ is a condition under which a conclusion can be inferred from hypotheses:
    \[
        \AxiomC{$\Gamma_0 \Rightarrow_S \varphi_0$}
        \AxiomC{$\dots$}
        \AxiomC{$\Gamma_n \Rightarrow_S \varphi_n$}  
        \TrinaryInfC{$\Delta \Rightarrow_S \psi$}
        \DisplayProof
    \]
    Presented with hypotheses above and conclusion below.

    \bigskip
    In HOL:
    \begin{alltt}
        S \(\Gamma_0\) \(\varphi_0\) /\bs{} ... /\bs{} S \(\Gamma_n\) \(\varphi_n\) ==> S \(\Delta\) \(\psi\)
    \end{alltt}

    In HOL, \texttt{S} is a prefix binary relation, instead of infix ($\Rightarrow_S$).
\end{frame}

% \begin{frame}
%     \frametitle{Derivations}
%     \begin{defn}
%          You can derive $\Delta$ from $\Gamma$ in $S$, 
%          if there are 1 or more applications of inference rules in $S$ which, 
%          start with $\Gamma$ as assumption(s) and end with $\Delta$ as conclusion(s).
%     \end{defn}
%     \begin{notn}
%         I write $\Gamma \vdash_S \Delta$. 
%     \end{notn}
% \end{frame}

\section{The Calculi}

\begin{frame}
    \frametitle{Natural Deduction}
    Natural Deduction ($N$) is a calculus which is thought to be somewhat `natural', in terms of the way we normally reason. Hypotheses in $N$ are a set of formulae. $N$ has introduction and elimination rules, as well as one axiom. Here are some of them:

    \[
        \AxiomC{}
        \RightLabel{(Axiom)}
        \UnaryInfC{$\{A\} \Rightarrow A$}
        \DisplayProof 
    \]
    \[
        \AxiomC{$\Gamma \Rightarrow B$}
        \RightLabel{$\to i$}
        \UnaryInfC{$\Gamma\setminus\{A\}\Rightarrow A\to B$}
        \DisplayProof
        \hspace{0.5cm}
        \AxiomC{$\Gamma\Rightarrow A\to B$}
        \AxiomC{$\Delta\Rightarrow A$}
        \RightLabel{$\to e$}
        \BinaryInfC{$\Gamma\cup\Delta\Rightarrow B$}
        \DisplayProof
    \]
\end{frame}
% \begin{frame}
%     \frametitle{Natural Deduction (cont.)}
%     In Hol:
%     \begin{alltt}
%         (! (A :'a formula).\ Nm \{A\} A)

%         /\char`\\\ (!A B D.\ (Nm (A INSERT D) B) ==> Nm D (A Imp B))

%         /\char`\\\ (!A B D1 D2.\ (Nm D1 (A Imp B)) /\char`\\\ (Nm D2 A)

%         ~ ==>\ Nm (D1 UNION D2) B)
%     \end{alltt}
%     \begin{notn}
%         I write \texttt{Nm} to indicate I am a minimal intuitionistic version of logic, without rules for $\bot$.
%     \end{notn}
% \end{frame}


\begin{frame}
    \frametitle{Sequent Calculus (aka. Gentzen System)}
    Sequent Calculus ($G$) is more mathematically elegant, but perhaps less intuitive than $N$. The hypotheses in $G$ are a multiset, i.e. $\{A,A\}\neq\{A\}$ as it would in a set, but order still doesn't matter like in a sequence. $G$ has Left and Right rules.

    Here are some of them:
    \[
        \AxiomC{}
        \RightLabel{(Axiom)}
        \UnaryInfC{$\Gamma \cupplus \{A\}\Rightarrow A$}
        \DisplayProof
    \]
    \[
        \AxiomC{$\Gamma\Rightarrow A$}
        \AxiomC{$\Gamma\cupplus\{B\}\Rightarrow C$}
        \RightLabel{L$\to$}
        \BinaryInfC{$\Gamma\cupplus\{A\to B\}\Rightarrow C$}
        \DisplayProof
        \hspace{0.5cm}
        \AxiomC{$\Gamma\cupplus\{A\}\Rightarrow B$}
        \RightLabel{R$\to$}
        \UnaryInfC{$\Gamma\Rightarrow A\to B$}
        \DisplayProof
    \]
\end{frame}

% \begin{frame}
%     \frametitle{Sequent Calculus (aka. Gentzen System)}
%     In HOL:
%     \begin{alltt}
%         (!(A:'a formula) \textGamma. (A <: \textGamma) /\char`\\\ FINITE_BAG \textGamma\\
%         ~ ==> Gm \textGamma A)\\
%         /\char`\\\ (!A B \textGamma C. (Gm \textGamma A) /\char`\\\ (Gm (BAG_INSERT B \textGamma) C)\\
%         ~ ==> (Gm (BAG_INSERT (A Imp B) \textGamma) C))\\
%         /\char`\\\ (!A B \textGamma. (Gm (BAG_INSERT A \textGamma) B)\\
%         ~ ==> (Gm \textGamma (A Imp B)))\\
%    \end{alltt}
%    \begin{notn}
%        In HOL, multisets are called bags.
%    \end{notn}
% \end{frame}

\section{The Proof}

\begin{frame}
    \frametitle{The Proof of Equivalence}
    \begin{itemize}
        \item For my project, I wanted to formalise the proof of the equivalence of $N$ and $G$, i.e.\ exactly the same conclusions can be derived from the same hypotheses. Since $N$ uses sets and $G$ uses multisets, this became a bit trickier than expected. 
        \item I was originally going to try to formalise the proofs for classical logic as well as intuitionistic logic, but so far I've only done the minimal intuitionistic version.
        \item I have had to prove many lemmata which are not provided in HOL (I plan to contribute some of them).
    \end{itemize}
\end{frame}

\begin{frame}[fragile]
    \frametitle{The Proof of Equivalence}
    Here is the statement in mathematical notation:
    \begin{thm}
        \[\forall ~\Gamma ~A. ~\Gamma \vdash_N A ~\Leftrightarrow~ \Gamma \vdash_G A\]
    \end{thm}
    In HOL:

    \begin{alltt}
        !\textGamma A. Gm \textGamma A <=> Nm (SET_OF_BAG \textGamma) A
    \end{alltt}
\end{frame}

\begin{frame}[fragile]
    \frametitle{Proof of $N\Rightarrow G$ in mathematics}
\begin{alltt}
Theorem Nm_Gm 
    `!\textGamma A. Nm \textGamma A ==> Gm (BAG_OF_SET \textGamma) A`
\end{alltt}
The proof is by rule induction (similar to structural induction). 

Modus Ponens case:
nice pic here
\[\]

\end{frame}

\begin{frame}[fragile] 
    \frametitle{Proof of $N\Rightarrow G$ in HOL}
    \small
\begin{alltt}
(simp[BAG_OF_SET_UNION] >>
`FINITE_BAG (BAG_OF_SET \textGamma')` 
  by metis_tac[Nm_FINITE,FINITE_BAG_OF_SET] >>
`Gm (BAG_INSERT A' (BAG_OF_SET \textGamma')) A'`
  by simp[Gm_ax,BAG_IN_BAG_INSERT] >>
`Gm (BAG_INSERT (A Imp A') (BAG_OF_SET \textGamma')) A'`
  by metis_tac[Gm_limp] >>
`Gm ((BAG_OF_SET \textGamma) \(\cupplus\) (BAG_OF_SET \textGamma')) A'`
  by metis_tac[Gm_cut] >>
`Gm (unibag (BAG_OF_SET \textGamma \(\cupplus\) BAG_OF_SET \textGamma')) A'` 
  by metis_tac[Gm_unibag] >>
fs[unibag_UNION])
\end{alltt}
\end{frame}

% \begin{frame}
%     % \frametitle{Proof of $N\Rightarrow G$ (cont.)}
%     Another goal is:
%     \texttt{Gm (BAG_OF_SET \textGamma) (A Imp A')}

%     With Hypotheses:\\
%     \texttt{0.  Nm (A INSERT \textGamma) A'\\
%          1.  Gm (BAG_OF_SET (A INSERT \textGamma)) A' }

%      Proved with:
%      \begin{alltt}
%          >- (irule Gm_rimp >>\\
%              fs[BAG_OF_SET_INSERT] >>\\
%              irule Gm_lw >>\\
%              simp[] >>\\
%              drule Gm_FINITE >>\\
%              rw[] >>\\
%              qexists_tac `BAG_MERGE \{|A|\} (BAG_OF_SET \textGamma)` >>\\
%              simp[BAG_MERGE_ELBAG_SUB_BAG_INSERT])
%      \end{alltt}
% \end{frame}

% \begin{frame}
%     \frametitle{Proof of $N\Rightarrow G$ (cont.)}
%     I won't go through any other parts of the proof, as it takes too long to explain. You get the idea...
% \end{frame}

\begin{frame}[fragile]{Proof of $G\Rightarrow N$}
    The proof in the other direction seems to require a subset of the hypotheses:
    \begin{alltt}
Theorem Gm_Nm 
`!\textGamma A. Gm \textGamma A ==> ?\textGamma'. \textGamma' \(\subseteq\) SET_OF_BAG \textGamma /\bs{} Nm \textGamma' A` 
    \end{alltt}
\end{frame}

\section{Conclusion}

\begin{frame}
    \frametitle{Conclusion}
    \begin{itemize}
        \item A lot of my time has been spent proving lemmata, mostly multiset/bag related.
        \item While it has been lot of work, I have learnt a lot about interactive theorem proving.
        \item I finished the proof for minimal intuitionistic logic.
        % \item I will still try to do improve upon what I've done formalising extended logics (maybe even classical!).
        \item My project will become an example within the distribution of HOL, and some of my lemmata will make it into the main distribution.
    \end{itemize}
\end{frame}

\end{document}
%%% Local Variables: 
%%% mode: latex
%%% TeX-PDF-mode: t 
%%% TeX-master: t
%%% End: 
