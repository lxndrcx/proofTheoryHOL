\documentclass[english,svgnames,hide notes,12pt]{beamer}
\usepackage{macros-ohp}
\usepackage{/Users/alexc/HOL/src/TeX/holtexbasic}
\usepackage{bussproofs}
\usepackage{alltt}
\usepackage{MnSymbol}
% \usepackage{listings}             

\theoremstyle{definition}
\newtheorem{defn}{Definition}
\theoremstyle{remark}
\newtheorem{notn}{Notation}

\def\presentationtitle{Proof of calculi equivalence in HOL4}

\title{\large\presentationtitle}
\author{Alexander Cox\\
		\small Supervised by Michael Norrish\\
		\small The Australian National University
	}
\date{\today}

\AtBeginSection[]{
  \begin{frame}
    \frametitle{Table of Contents}
    \tableofcontents[currentsection]
  \end{frame}}

\begin{document}
\thispagestyle{empty}
%\begin{frame}[plain] %put [plain] at the end to get rid of the page number on this page
\begin{frame}
    \titlepage{}
\end{frame}

\begin{frame}
    \frametitle{Introduction}
    For my VRS project, I have been formalising some proof theory in an interactive theorem prover called HOL. 

    The proof I have been formalising is the equivalence between Natural Deduction and Sequent Calculus, for intuitionistic propositional logic.
\end{frame}

\section{Theorem Proving in HOL}

\begin{frame}
    \frametitle{Theorem Provers}
    \begin{itemize}
        \item A \emph{Interactive Theorem Prover} or \emph{Proof Assistant} is software used to formalise and verify the correctness of proofs.
        \item Proving a proposition in a proof assistant usually takes significantly more time than proving an informal proof on paper.
    \end{itemize}
\end{frame}

\begin{frame}
    \frametitle{Why Bother?}
    \begin{enumerate}
        \item Once a theorem is proved in a proof assistant, you can trust that it is a correct proof, provided that the proof assistant itself is sound.
        \item Proving a theorem formally can reveal problems with the informal proof in some cases, since details cannot be skimmed over like they sometimes are in informal proofs.
        \item It is also fun (in my opinion).
    \end{enumerate}
\end{frame}

\begin{frame}
	\frametitle{The HOL Theorem Prover}
    \begin{itemize}
        \item The HOL Theorem Prover is a proof assistant developed since the mid-80s. 
        \item The version I have been using is called HOL4. 
        \item HOL implements a Higher Order Logic which is a variant of Church's simple theory of types.
        \item HOL is written in Standard ML (SML), which is also the language the user interacts with.
    \end{itemize}
\end{frame}

\begin{frame}
    \frametitle{Theorems in HOL}
    \begin{itemize}
        \item Theorems in HOL are represented as an SML datatype, and the only way to produce them is to use sound inference rules on predefined axioms.
        \item As a consequence, you can't prove something which is false, unless there is a fault with an axiom or inference rule.
        \item HOL is heavily scrutinised by experts in the field, and is generally trusted to be sound.
    \end{itemize}
\end{frame}

\begin{frame}
	\frametitle{Goal-directed Proof}
    \begin{itemize}
        \item When proving something in HOL, I have been using goal-directed proofs.
        \item I state the goal I want to prove, and try to prove it using \emph{tactics}. Some tactics are very powerful and can prove simple goals without much help at all. Others goals require more work, often splitting into multiple sub-goals. 
        \item Generally I work forwards (from assumptions) and backwards (from the conclusion/goal) at the same time.
    \end{itemize}
\end{frame}

\section{Proof Theory and Logic}

\begin{frame}
    \frametitle{Proof Theory}
    \begin{itemize}
        \item Proof theory is a branch of mathematical logic which analyses proofs and their calculi. 
        \item Here the focus is on the syntax of a logic, rather than it's semantics.
    \end{itemize}
\end{frame}

\begin{frame}
    % \frametitle{Propositional Logic Syntax}
    \begin{defn}
        A propositional logic formula $\varphi$ is defined inductively:
        \[ \varphi ::= a ~|~ \varphi \lor \varphi ~|~ \varphi \land \varphi ~|~ \varphi \to \varphi ~|~ \bot \]
    In HOL:
    \begin{alltt}
        val _ = Datatype `formula =

        Var 'a
        
        | Or formula formula
        
        | And formula formula
        
        | Imp formula formula
        
        | Bot`;
    \end{alltt}
    \end{defn}
\end{frame}
\begin{frame}
    % \frametitle{Propositional Logic Syntax (cont.)}
    \begin{notn}
    Some abbreviations: 
        \[ \neg \varphi := \varphi \to \bot \]
        \[ \varphi \Leftrightarrow \psi := \varphi \to \psi \land \psi \to \varphi \]
        \[ \top := \bot \to \bot \]
        In HOL:
        \begin{alltt}
            val Not_def = Define `Not f = f Imp Bot`;

            val BiImp_def = 
            
            ~Define `f BiImp f' = (f Imp f') And (f' Imp f)`;

            val Top_def = Define `Top = Bot Imp Bot`;
        \end{alltt}
    \end{notn}
\end{frame}

\begin{frame}
    \frametitle{Inference Rules}
    \begin{defn}
        An inference rule in a system $S$ is a rule:
        \[
            \AxiomC{$\Gamma_0 \Rightarrow_S \varphi_0$}
            \AxiomC{$\dots$}
            \AxiomC{$\Gamma_n \Rightarrow_S \varphi_n$}  
            \TrinaryInfC{$\Delta \Rightarrow_S \psi$}
            \DisplayProof
        \]
        Presented with hypotheses above and conclusion below.
        In HOL:
        \begin{alltt}
            % !$\Gamma_0$ ... $\Gamma_n$ $\Delta$ $\varphi_0$ ... $\varphi_n$ $\psi$. 
            S $\Gamma_0 ~ \varphi_0$ /\char`\\\ ... /\char`\\\ S $\Gamma_n ~ \varphi_n$ 
            ==> S $\Delta ~ \psi$
        \end{alltt}
    \end{defn}
\end{frame}

\begin{frame}
    \frametitle{Derivations}
    \begin{defn}
         You can derive $\Delta$ from $\Gamma$ in $S$, 
         if there are 1 or more applications of inference rules in $S$ which, 
         start with $\Gamma$ as assumption(s) and end with $\Delta$ as conclusion(s).
    \end{defn}
    \begin{notn}
        I write $\Gamma \vdash_S \Delta$. 
    \end{notn}
\end{frame}

\section{The Calculi}

\begin{frame}
    \frametitle{Natural Deduction}
    Natural Deduction ($N$) is a calculus which is thought to be somewhat `natural', in terms of the way we normally reason. Hypotheses in $N$ are a set of formulae. $N$ has introduction and elimination rules, as well as one axiom. Here are some of them:

    \[
        \AxiomC{}
        \RightLabel{(Axiom)}
        \UnaryInfC{$\{A\} \Rightarrow A$}
        \DisplayProof 
    \]
    \[
        \AxiomC{$\Gamma\setminus\{A\} \Rightarrow B$}
        \RightLabel{$\to i$}
        \UnaryInfC{$\Gamma\Rightarrow A\to B$}
        \DisplayProof
        \hspace{0.5cm}
        \AxiomC{$\Gamma\Rightarrow A\to B$}
        \AxiomC{$\Delta\Rightarrow A$}
        \RightLabel{$\to e$}
        \BinaryInfC{$\Gamma\cup\Delta\Rightarrow B$}
        \DisplayProof
    \]
\end{frame}
\begin{frame}
    \frametitle{Natural Deduction (cont.)}
    In Hol:
    \begin{alltt}
        (! (A :'a formula).\ Nm \{A\} A)

        /\char`\\\ (!A B D.\ (Nm (A INSERT D) B) ==> Nm D (A Imp B))

        /\char`\\\ (!A B D1 D2.\ (Nm D1 (A Imp B)) /\char`\\\ (Nm D2 A)

        ~ ==>\ Nm (D1 UNION D2) B)
    \end{alltt}
    \begin{notn}
        I write \texttt{Nm} to indicate I am a minimal intuitionistic version of logic, without rules for $\bot$.
    \end{notn}
\end{frame}


\begin{frame}
    \frametitle{Sequent Calculus (aka. Gentzen System)}
    Sequent Calculus ($G$) is supposedly more mathematically elegant, but perhaps somewhat less intuitive than $N$. The hypotheses in $G$ are a multiset, i.e. $\{A,A\}\neq\{A\}$ as it would in a set, but order still doesn't matter like in a sequence. $G$ has Left and Right rules.

    Here are some of them:
    \[
        \AxiomC{}
        \RightLabel{(Axiom)}
        \UnaryInfC{$\Gamma \cupplus \{A\}\Rightarrow A$}
        \DisplayProof
    \]
    \[
        \AxiomC{$\Gamma\Rightarrow A$}
        \AxiomC{$\Gamma\cupplus\{B\}\Rightarrow C$}
        \RightLabel{L$\to$}
        \BinaryInfC{$\Gamma\cupplus\{A\to B\}\Rightarrow C$}
        \DisplayProof
        \hspace{0.5cm}
        \AxiomC{$\Gamma\cupplus\{A\}\Rightarrow B$}
        \RightLabel{R$\to$}
        \UnaryInfC{$\Gamma\Rightarrow A\to B$}
        \DisplayProof
    \]
\end{frame}

\begin{frame}
    \frametitle{Sequent Calculus (aka. Gentzen System)}
    In HOL:
    \begin{alltt}
    \end{alltt}
\end{frame}

\end{document}
%%% Local Variables: 
%%% mode: latex
%%% TeX-PDF-mode: t 
%%% TeX-master: t
%%% End: 
