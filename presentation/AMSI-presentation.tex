\documentclass[english,svgnames,hide notes,12pt]{beamer}
\usepackage{macros-ohp}
\usepackage{/Users/alexc/HOL/src/TeX/holtexbasic}
\usepackage{bussproofs}
\usepackage{alltt}
\usepackage{MnSymbol}
\usepackage{textgreek}
\usepackage[backend=biber, style=numeric-comp]{biblatex}
\addbibresource{presentation.bib}

\newtheorem{thm}{Theorem}
\theoremstyle{definition}
\newtheorem{defn}{Definition}
\theoremstyle{remark}
\newtheorem{notn}{Notation}

\def\presentationtitle{Proof of calculi equivalence in HOL4}

\title{\large\presentationtitle}
\author{Alexander Cox\\
		\small Supervised by Michael Norrish\\
		\small The Australian National University
	}
\date{\today}

% \AtBeginSection[]{
%   \begin{frame}
%     \frametitle{Table of Contents}
%     \tableofcontents[currentsection]
%   \end{frame}}

\begin{document}
\thispagestyle{empty}
\begin{frame}
    \titlepage{}
\end{frame}

\begin{frame}
    % \frametitle{Introduction}
    Over the summer, I have been formalising some logic in an interactive theorem prover called HOL.

    \bigskip
    The proof I have been formalising is the equivalence between Natural Deduction and Sequent Calculus, for intuitionistic propositional logic.

    \bigskip
    This result comes from \emph{proof theory}, a branch of mathematical logic that analyses proofs and their calculi.

    \bigskip
    I have been working from \fullcite{bpt}.
\end{frame}

\section{Theorem Proving in HOL}

\begin{frame}
    \frametitle{Theorem Provers}
    \begin{itemize}
        \item An \emph{Interactive Theorem Prover} or \emph{Proof Assistant} is software used to formalise and verify the correctness of proofs.
        \item The prover I used (HOL4) is engineered so that theorems can only be produced under the control of a small trusted kernel.
        \item Proving a proposition in a theorem prover usually takes significantly more time than proving an informal proof on paper.
    \end{itemize}
\end{frame}

\begin{frame}
    \frametitle{Why Bother?}
    \begin{enumerate}
        \item Once a theorem is proved in a theorem prover, you can trust that it is a correct proof, provided that the theorem prover itself is sound.
        \item Proving a theorem formally can reveal problems with the informal proof in some cases, since details cannot be skimmed over like they sometimes are in informal proofs.
        \item I wanted to learn about theorem provers, and this seemed like a good project for that.
    \end{enumerate}
\end{frame}

\section{Proof Theory and Logic}

\begin{frame}
    \frametitle{Proofs about proofs}
    I have been talking about proofs in HOL, which could be based in any kind of mathematics, e.g., number theory, analysis, programming languages, etc.

    \bigskip
    Now I'm going to talk about proof theory in HOL, i.e., proofs about proofs in HOL. HOL is the meta-logic used to formalise the logic which is used in the proof systems I am analysing.
\end{frame}

\begin{frame}[fragile]
    % \frametitle{Propositional Logic Syntax}
    Proofs manipulate logic formulae, which are defined inductively:
    \[ \varphi ::= a~|~\varphi \lor \varphi~|~\varphi \land \varphi~|~\varphi \to \varphi~|~\bot \]

    \bigskip
    In HOL:
    \small
\begin{alltt}
val _ = Datatype `formula =
    Var 'a
    | Or formula formula
    | And formula formula
    | Imp formula formula
    | Bot`;
\end{alltt}
\end{frame}

\newcommand{\bs}{\char`\\}
\begin{frame}[fragile]
    \frametitle{Inference Rules}
    An inference rule in a system $S$ is a condition under which a conclusion can be inferred from hypotheses:
    \[
        \AxiomC{$\Gamma_0 \vdash_S \varphi_0$}
        \AxiomC{$\dots$}
        \AxiomC{$\Gamma_n \vdash_S \varphi_n$}
        \RightLabel{(rule)}
        \TrinaryInfC{$\Delta \vdash_S \psi$}
        \DisplayProof
    \]
    Presented with hypotheses above and conclusion below.

    \bigskip
    In HOL:
    \begin{alltt}
        S \(\Gamma_0\) \(\varphi_0\) /\bs{} ... /\bs{} S \(\Gamma_n\) \(\varphi_n\) ==> S \(\Delta\) \(\psi\)
    \end{alltt}

    In HOL, \texttt{S} is a prefix binary relation, instead of infix ($\vdash_S$).
\end{frame}

\section{The Calculi}

\begin{frame}
    \frametitle{Natural Deduction}
    Natural Deduction ($N$) is a calculus which is thought to be somewhat `natural', in terms of the way we normally reason. Hypotheses in $N$ are a set of formulae. $N$ has introduction and elimination rules, as well as one axiom. Here are some of them:

    \[
        \AxiomC{}
        \RightLabel{(Axiom)}
        \UnaryInfC{$\{A\} \vdash_N A$}
        \DisplayProof
        \hspace{0.5cm}
        \AxiomC{$\Gamma \vdash_N B$}
        \RightLabel{$\to i$}
        \UnaryInfC{$\Gamma\setminus\{A\}\vdash_N A\to B$}
        \DisplayProof
    \]
    The following elimination rule corresponds to modus ponens:
    \[
        \AxiomC{$\Gamma\vdash_N A\to B$}
        \AxiomC{$\Delta\vdash_N A$}
        \RightLabel{$\to e$}
        \BinaryInfC{$\Gamma\cup\Delta\vdash_N B$}
        \DisplayProof
    \]
\end{frame}


\begin{frame}
    \frametitle{Sequent Calculus (aka. Gentzen System)}
    Sequent Calculus ($G$) is more mathematically elegant, but perhaps less intuitive than $N$. The hypotheses in $G$ are a multiset, i.e., $\{A,A\}\neq\{A\}$ as it would be in a set, but order still doesn't matter like in a sequence. $G$ has Left and Right rules, and one axiom.

    Here are some of them:
    \[
        \AxiomC{}
        \RightLabel{(Axiom)}
        \UnaryInfC{$\Gamma \cupplus \{A\}\vdash_G A$}
        \DisplayProof
    \]
    \[
        \AxiomC{$\Gamma\vdash_G A$}
        \AxiomC{$\Gamma\cupplus\{B\}\vdash_G C$}
        \RightLabel{L$\to$}
        \BinaryInfC{$\Gamma\cupplus\{A\to B\}\vdash_G C$}
        \DisplayProof
        \hspace{0.5cm}
        \AxiomC{$\Gamma\cupplus\{A\}\vdash_G B$}
        \RightLabel{R$\to$}
        \UnaryInfC{$\Gamma\vdash_G A\to B$}
        \DisplayProof
    \]
\end{frame}

\section{The Proof}

\begin{frame}
    \frametitle{The Proof of Equivalence}
    \begin{itemize}
        \item For my project, I wanted to formalise the proof of the equivalence of $N$ and $G$, i.e.,\ exactly the same conclusions can be derived from the same hypotheses in both systems. Since $N$ uses sets and $G$ uses multisets, this has been trickier than expected.
        \item I was originally going to try to formalise the proofs for classical logic as well as intuitionistic logic, but so far I've only done the minimal intuitionistic version.
        \item I have had to prove many lemmata which are not provided in HOL (I plan to contribute some of them).
    \end{itemize}
\end{frame}

\begin{frame}[fragile]
    \frametitle{The Proof of Equivalence}
    Here is the statement in mathematical notation:

    (read $\Gamma\vdash_S A$ as ``$\Gamma$ derives $A$ in $S$'')
    \begin{thm}
        \[\forall ~\Gamma ~A. ~\Gamma \vdash_N A ~\Leftrightarrow~ \Gamma \vdash_G A\]
    \end{thm}
    In HOL:

    \begin{alltt}
        \(\forall\) \textGamma A. Gm \textGamma A <=> Nm (SET_OF_BAG \textGamma) A
    \end{alltt}
\end{frame}

\begin{frame}[fragile]
    \frametitle{Proof of $N\Rightarrow G$}
\small
\begin{alltt}
Theorem Nm_Gm
    `\(\forall\) \textGamma A. Nm \textGamma A ==> Gm (BAG_OF_SET \textGamma) A`
\end{alltt}
The proof is by rule induction. Each case corresponds to a rule in $N$.

Modus Ponens ($\to$e) case:
\[
    \AxiomC{}
    \RightLabel{IH}
    \UnaryInfC{$\Gamma_0\vdash_N A\to B$}
    \AxiomC{}
    \RightLabel{IH}
    \UnaryInfC{$\Gamma_1\vdash_N A$}
    \RightLabel{$\to$e}
    \BinaryInfC{$\Gamma_0\cup\Gamma_1 \vdash_N B$}
    \DisplayProof
\]
becomes
\[
    \AxiomC{}
    \RightLabel{IH}
    \UnaryInfC{$\Gamma_0 \vdash_G A \to B$}
    \AxiomC{}
    \RightLabel{IH}
    \UnaryInfC{$\Gamma_1 \vdash_G A$}
    \AxiomC{}
    \RightLabel{axiom}
    \UnaryInfC{$\{B\} \vdash_G B$}
    \RightLabel{L$\to$}
    \BinaryInfC{$\{A\to B\}\cupplus\Gamma_1\vdash_G B$}
    \RightLabel{cut}
    \BinaryInfC{$\Gamma_0\cupplus\Gamma_1\vdash_G B$}
    \RightLabel{contraction}
    \UnaryInfC{$\text{set}(\Gamma_0\cupplus\Gamma_1)\vdash_G B$}
    \DisplayProof
\]

\end{frame}

\begin{frame}[fragile]
    \frametitle{Proof of $N\Rightarrow G$ in HOL}
    \small
\begin{alltt}
(simp[BAG_OF_SET_UNION] >>
`FINITE_BAG (BAG_OF_SET \textGamma')`
  by metis_tac[Nm_FINITE,FINITE_BAG_OF_SET] >>
`Gm (BAG_INSERT A' (BAG_OF_SET \textGamma')) A'`
  by simp[Gm_ax,BAG_IN_BAG_INSERT] >>
`Gm (BAG_INSERT (A Imp A') (BAG_OF_SET \textGamma')) A'`
  by metis_tac[Gm_limp] >>
`Gm ((BAG_OF_SET \textGamma) \(\cupplus\) (BAG_OF_SET \textGamma')) A'`
  by metis_tac[Gm_cut] >>
`Gm (unibag (BAG_OF_SET \textGamma \(\cupplus\) BAG_OF_SET \textGamma')) A'`
  by metis_tac[Gm_unibag] >>
fs[unibag_UNION])
\end{alltt}
\end{frame}

\begin{frame}[fragile]{Proof of $G\Rightarrow N$}
    The proof in the other direction seems to require a subset of the hypotheses:
    \begin{alltt}
Theorem Gm_Nm
`\(\forall\) \textGamma A. Gm \textGamma A ==> \(\exists\) \textGamma'. \textGamma' \(\subseteq\) SET_OF_BAG \textGamma /\bs{} Nm \textGamma' A`
    \end{alltt}
    % \includegraphics[width=\textwidth]{proof-complete-gn}
\end{frame}

\section{Conclusion}

\begin{frame}
    \frametitle{Conclusion}
    \begin{itemize}
        \item A lot of my time has been spent proving lemmata, mostly multiset/bag related.
        \item While it has been lot of work, I have learnt a lot about interactive theorem proving.
        \item I finished the proof for minimal intuitionistic logic.
        % \item I will still try to do improve upon what I've done formalising extended logics (maybe even classical!).
        \item My project will become an example within the distribution of HOL, and some of my lemmata will make it into the main distribution.
        \item The code for my project can be found at \url{https://github.com/lxndrcx/proofTheoryHOL}
    \end{itemize}
\end{frame}

\end{document}
%%% Local Variables:
%%% mode: latex
%%% TeX-PDF-mode: t
%%% TeX-master: t
%%% End:
