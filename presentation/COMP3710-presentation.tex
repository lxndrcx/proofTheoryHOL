\documentclass[english,svgnames,hide notes,12pt]{beamer}
% \usetheme{Madrid}
\useoutertheme{infolines}
\useinnertheme[shadow]{rounded}
\usecolortheme{seahorse}


\usepackage{bussproofs}
\usepackage{/Users/alexc/HOL/src/TeX/holtexbasic}
\usepackage{amsthm}
\usepackage{mathtools}
\usepackage{alltt}
\usepackage{MnSymbol}
\usepackage{textgreek}
\usepackage[backend=biber, style=numeric-comp]{biblatex}
\addbibresource{presentation.bib}

\newtheorem{thm}{Theorem}
\theoremstyle{definition}
\newtheorem{defn}{Definition}
\theoremstyle{remark}
\newtheorem{notn}{Notation}

\title[Equivalence of calculi in HOL4]{Equivalence of propositional logic proof calculi, formalised in HOL4}
\subtitle{COMP3710 Project, Bachelor of Science}

\author[Alexander Cox]{\large Alexander Cox\\ \small Supervised by Michael Norrish}
\institute[ANU]{The Australian National University}
\date[Summer 2019]{Summer Session 2019}

\begin{document}
\begin{frame}[plain]
    \titlepage{}
\end{frame}

\section{Introduction}

\begin{frame}
    \frametitle{Introduction}
    Over the summer, I have been formalising some proof theory in the HOL4 theorem prover.

    \bigskip
    The proof I have been formalising is the equivalence between Natural Deduction and Sequent Calculus, for intuitionistic propositional logic.

    \bigskip
    I have been working from \citetitle{bpt} by \citeauthor{bpt}, with some small deviations which I shall explain along the way.
\end{frame}

\section{Theorem Proving in HOL}

\begin{frame}{Theorem Provers}
    \begin{itemize}
        \item An \emph{Interactive Theorem Prover} or \emph{Proof Assistant} is software used to formalise and verify the correctness of proofs.
        \item Proving a proposition in a theorem prover usually takes significantly more time than proving an informal proof on paper.
    \end{itemize}
\end{frame}

\begin{frame}{The HOL Theorem Prover}
  I have been using the HOL theorem prover (HOL4), a theorem prover which implements a Higher Order Logic which is a variant of Church's Simple Theory of Types.
  HOL is engineered so that theorems can only be produced under the control of a small trusted kernel.
\end{frame}

\begin{frame}{Why Bother?}
    \begin{enumerate}
        \item Once a theorem is proved in a theorem prover, you can trust that it is a correct proof, provided that the theorem prover itself is sound.
        \item Proving a theorem formally can reveal problems with the informal proof in some cases, since details cannot be skimmed over like they sometimes are in informal proofs.
        \item I wanted to learn about theorem provers, and this seemed like a good project for that.
    \end{enumerate}
\end{frame}

\section{Proof Theory and Logic}

\begin{frame}{Proofs about proofs}
    I have been talking about proofs in HOL, now I'm going to talk about proof theory in HOL, i.e., proofs about proofs in HOL.

    \bigskip
    HOL is the meta-logic used to formalise the (object-)logic which is used in the proof systems I am analysing.

    \bigskip
    Standard ML is the meta-language of HOL and the interface to the logic.

    \bigskip
    English is the meta-language of this talk.
\end{frame}

\begin{frame}[fragile]
    \frametitle{Propositional Logic Syntax}
    Propositional logic formulae are defined inductively as follows:
    \[ \varphi ::= a~|~\varphi \lor \varphi~|~\varphi \land \varphi~|~\varphi \to \varphi~|~\bot \]

    \bigskip
    In HOL:
    \small
\begin{alltt}
val _ = Datatype `formula =
    Var 'a
    | Or formula formula
    | And formula formula
    | Imp formula formula
    | Bot`;
\end{alltt}
\end{frame}

\begin{frame}{Abbreviations}
  The following abbreviations are made:
  \[ \neg \varphi := \varphi \to \bot \]
  \[ \varphi \leftrightarrow \psi := (\varphi \to \psi) \land (\psi \to \varphi) \]
  \[ \top := \bot \to \bot \]
\end{frame}

\newcommand{\bs}{\char`\\}
\begin{frame}[fragile]
    \frametitle{Inference Rules}
    An inference rule in a system $S$ is a condition under which a conclusion can be inferred from hypotheses:
    \[
        \AxiomC{$\Gamma_0 \vdash_S \varphi_0$}
        \AxiomC{$\dots$}
        \AxiomC{$\Gamma_n \vdash_S \varphi_n$}
        \RightLabel{(rule)}
        \TrinaryInfC{$\Delta \vdash_S \psi$}
        \DisplayProof
    \]
    Presented with hypotheses above and conclusion below.

    \bigskip
    In HOL:
    \begin{alltt}
        S \(\Gamma_0\) \(\varphi_0\) /\bs{} ... /\bs{} S \(\Gamma_n\) \(\varphi_n\) ==> S \(\Delta\) \(\psi\)
    \end{alltt}

    In HOL, I represent the infix binary relation ($\vdash_S$) as a prefix \texttt{S} relation, for no particular reason.
\end{frame}

\section{The Calculi}

\begin{frame}
    \frametitle{Natural Deduction}
    In Natural Deduction ($N$) the hypotheses are a represented as a set of formulae. In the book, the hypotheses are labelled. In my formalisation I keep them unlabelled (Complete Discharge Convention, permissible). $N$ has introduction and elimination rules, as well as one axiom. Here are some of them:

    % TODO: my discharge is different from book but I prove it's ok.

    \[
        \AxiomC{}
        \RightLabel{(Axiom)}
        \UnaryInfC{$\{A\} \vdash_N A$}
        \DisplayProof
        \hspace{0.5cm}
        \AxiomC{$\Gamma \vdash_N B$}
        \RightLabel{$\to i$}
        \UnaryInfC{$\Gamma\setminus\{A\}\vdash_N A\to B$}
        \DisplayProof
    \]
    \[
        \AxiomC{$\Gamma\vdash_N A\to B$}
        \AxiomC{$\Delta\vdash_N A$}
        \RightLabel{$\to e$}
        \BinaryInfC{$\Gamma\cup\Delta\vdash_N B$}
        \DisplayProof
    \]
\end{frame}


\begin{frame}
    \frametitle{Sequent Calculus (aka. Gentzen System)}
    In Sequent Calculus ($G$) the hypotheses are represented as a multiset (or \emph{bag}). $G$ has Left and Right rules, and one axiom.

    Here are some of them:
    \[
        \AxiomC{}
        \RightLabel{(Axiom)}
        \UnaryInfC{$\Gamma \cupplus \{A\}\vdash_G A$}
        \DisplayProof
    \]
    \[
        \AxiomC{$\Gamma\vdash_G A$}
        \AxiomC{$\Gamma\cupplus\{B\}\vdash_G C$}
        \RightLabel{L$\to$}
        \BinaryInfC{$\Gamma\cupplus\{A\to B\}\vdash_G C$}
        \DisplayProof
        \hspace{0.5cm}
        \AxiomC{$\Gamma\cupplus\{A\}\vdash_G B$}
        \RightLabel{R$\to$}
        \UnaryInfC{$\Gamma\vdash_G A\to B$}
        \DisplayProof
    \]
\end{frame}

\section{The Proof}

\begin{frame}
    \frametitle{The Proof of Equivalence}
    \begin{itemize}
        \item For my project, I wanted to formalise the proof of the equivalence of $N$ and $G$, i.e.,\ exactly the same conclusions can be derived from the same hypotheses in both systems. Since $N$ uses sets and $G$ uses multisets, this has been trickier than expected.
        \item I was originally going to try to formalise the proofs for classical logic as well as intuitionistic logic, but I've only done the intuitionistic version.
        \item I have had to prove many lemmata which are not provided in HOL.
    \end{itemize}
\end{frame}

\begin{frame}[fragile]
    \frametitle{The Proof of Equivalence}
    Here is the statement in mathematical notation:

    (read $\Gamma\vdash_S A$ as ``$\Gamma$ derives $A$ in $S$'')
    \begin{thm}
        \[\forall ~\Gamma ~A. ~\Gamma \vdash_N A ~\Leftrightarrow~ \Gamma \vdash_G A\]
    \end{thm}
    In HOL:

    \begin{alltt}
        \(\forall\) \textGamma A. Gm \textGamma A <=> Nm (SET_OF_BAG \textGamma) A
    \end{alltt}
\end{frame}

\begin{frame}[fragile]
    \frametitle{Proof of $N\Rightarrow G$}
\small
\begin{alltt}
Theorem Nm_Gm
    `\(\forall\) \textGamma A. Nm \textGamma A ==> Gm (BAG_OF_SET \textGamma) A`
\end{alltt}
The proof is by rule induction. Each case corresponds to a rule in $N$.

Modus Ponens ($\to$e) case:
\[
    \AxiomC{}
    \RightLabel{IH}
    \UnaryInfC{$\Gamma_0\vdash_N A\to B$}
    \AxiomC{}
    \RightLabel{IH}
    \UnaryInfC{$\Gamma_1\vdash_N A$}
    \RightLabel{$\to$e}
    \BinaryInfC{$\Gamma_0\cup\Gamma_1 \vdash_N B$}
    \DisplayProof
\]
becomes
\[
    \AxiomC{}
    \RightLabel{IH}
    \UnaryInfC{$\Gamma_0 \vdash_G A \to B$}
    \AxiomC{}
    \RightLabel{IH}
    \UnaryInfC{$\Gamma_1 \vdash_G A$}
    \AxiomC{}
    \RightLabel{axiom}
    \UnaryInfC{$\{B\} \vdash_G B$}
    \RightLabel{L$\to$}
    \BinaryInfC{$\{A\to B\}\cupplus\Gamma_1\vdash_G B$}
    \RightLabel{cut}
    \BinaryInfC{$\Gamma_0\cupplus\Gamma_1\vdash_G B$}
    \RightLabel{contraction}
    \UnaryInfC{$\text{set}(\Gamma_0\cupplus\Gamma_1)\vdash_G B$}
    \DisplayProof
\]

\end{frame}

\begin{frame}[fragile]
    \frametitle{Proof of $N\Rightarrow G$ in HOL}
    \small
\begin{alltt}
(simp[BAG_OF_SET_UNION] >>
`FINITE_BAG (BAG_OF_SET \textGamma')`
  by metis_tac[Nm_FINITE,FINITE_BAG_OF_SET] >>
`Gm (BAG_INSERT A' (BAG_OF_SET \textGamma')) A'`
  by simp[Gm_ax,BAG_IN_BAG_INSERT] >>
`Gm (BAG_INSERT (A Imp A') (BAG_OF_SET \textGamma')) A'`
  by metis_tac[Gm_limp] >>
`Gm ((BAG_OF_SET \textGamma) \(\cupplus\) (BAG_OF_SET \textGamma')) A'`
  by metis_tac[Gm_cut] >>
`Gm (unibag (BAG_OF_SET \textGamma \(\cupplus\) BAG_OF_SET \textGamma')) A'`
  by metis_tac[Gm_unibag] >>
fs[unibag_UNION])
\end{alltt}
\end{frame}

\begin{frame}[fragile]{Proof of $G\Rightarrow N$}
    The proof in the other direction seems to require a subset of the hypotheses:
    \begin{alltt}
Theorem Gm_Nm
`\(\forall\) \textGamma A. Gm \textGamma A ==> \(\exists\) \textGamma'. \textGamma' \(\subseteq\) SET_OF_BAG \textGamma /\bs{} Nm \textGamma' A`
    \end{alltt}
\end{frame}

\section{Conclusion}

\begin{frame}
    \frametitle{Conclusion}
    \begin{itemize}
        \item A lot of my time has been spent proving lemmata, mostly multiset/bag related.
        \item While it has been lot of work, I have learnt a lot about interactive theorem proving.
        \item I finished the proof for minimal and intuitionistic logic.
        \item My project may become an example within the distribution of HOL.
        \item Some of my bag lemmata have been merged into HOL, see \url{https://github.com/HOL-Theorem-Prover/HOL/pull/654}
        \item The code, report and this seminar can be found at \url{https://github.com/lxndrcx/proofTheoryHOL}
    \end{itemize}
\end{frame}

\end{document}
%%% Local Variables:
%%% mode: latex
%%% TeX-PDF-mode: t
%%% TeX-master: t
%%% End:
